% !TEX encoding = UTF-8 Unicode
% !TEX root = main.tex

\documentclass[sfsidenotes,notoc,nobib,a4paper]{tufte-book}

\title{Book example}
\author{Humberto Stein Shiromoto}
\date{\today}


\begin{document}
\setcounter{tocdepth}{1}
\maketitle

% v.2 epigraphs
% !TEX encoding = UTF-8 Unicode
% !TEX root = main.tex


\newpage\thispagestyle{empty}
\openepigraph{%
The public is more familiar with bad design than good design.
It is, in effect, conditioned to prefer bad design, 
because that is what it lives with. 
The new becomes threatening, the old reassuring.
}{Paul Rand, {\itshape Design, Form, and Chaos}
}
\vfill
\openepigraph{%
La perfection est atteinte, non pas lorsqu\'il n'y a plus rien \`a  ajouter, mais lorsqu\'il n'y a plus rien \`a  retirer.
}{Antoine de Saint-Exup\'{e}ry}
\vfill

% v.4 copyright page
% !TEX encoding = UTF-8 Unicode
% !TEX root = main.tex


\newpage
\begin{fullwidth}
~\vfill
\thispagestyle{empty}
\setlength{\parindent}{0pt}
\setlength{\parskip}{\baselineskip}
Copyright \copyright\ \the\year\ \thanklessauthor

\par\smallcaps{Published by \thanklesspublisher}

\par\smallcaps{humberto.stein-shiromoto.info}

\par Licensed under the Apache License, Version 2.0 (the ``License''); you may not
use this file except in compliance with the License. You may obtain a copy
of the License at \url{http://www.apache.org/licenses/LICENSE-2.0}. Unless
required by applicable law or agreed to in writing, software distributed
under the License is distributed on an \smallcaps{``AS IS'' BASIS, WITHOUT
WARRANTIES OR CONDITIONS OF ANY KIND}, either express or implied. See the
License for the specific language governing permissions and limitations
under the License.\index{license}

\par\textit{First printing, \monthyear}
\end{fullwidth}
\tableofcontents
\adjustmtc
\printnomenclature

\appendix
%\input{appendix.tex}

% !TEX encoding = UTF-8 Unicode
% !TEX root = main.tex

\chapter{Bibliography}
\minitoc

\begin{description}
	\item \citep{Liberzon2012} is a comprehensive book on the optimisation. It starts the book by introducing finite and infinite-dimensional optimisation problems. The next subject is the calculus of variations, and optimal control.
\end{description}


\clearpage
\printbibliography[segment=\therefsegment,heading=subbibintoc]

\printbibliography[heading=bibintoc]
\printindex

\end{document}

% Remarks
% 2) To compile the index use
% 2a) XeLaTeX main.tex
% 2b) makeindex main.idx
% 2c) XeLaTeX main.tex
%
% 3) To compile the nomenclature use
% 3a) XeLaTeX main.tex
% 3b) makeindex main.nlo -s nomencl.ist -o main.nls
% 3c) XeLaTeX main.tex
%
% 4) To compile the bibliography use
% 4a) XeLaTeX main.tex
% 4b) Biber mainj-blx.aux, with j = 1,2,...,n (number of refsection environments)
% 4c) XeLaTeX main.tex

% To use reverse sync with sumatra PDF and sublime text 3:
% If you have a PDF file with a corresponding .synctex.gz file, then open it in Sumatra and go to Settings | Options, and enter:

% "C:\Program Files\Sublime Text 3\sublime_text.exe" "%f:%l"